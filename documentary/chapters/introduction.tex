Multicore systems are becoming increasingly popular as part of digitization and Industry 4.0\footnote{ add.: https://www.epicor.com/en-ae/resource-center/articles/what-is-industry-4-0/} (German/EU) \parencite{internet5} \parencite[or][]{article13} - also known as smart manufacoring in the USA \parencite[see][p1]{article12} - and are playing an important role in data processing and process automation \parencite[see][p294]{inproceedings3} \parencite[or][p1]{article11}. On the other hand, in addition to efficiency in energy consumption, performance in terms of computation time \parencite[see][p294]{inproceedings3} is required in almost every application field of multicore systems.

In fact, multicore hardware is not only exspecially for smart manufactoring. Nowerdays in almost every smart application like smart phones\footnote{ e.g. ARM based processors for mobile phones like https://www.arm.com/solutions/mobile-computing/smartphones}, wearables\footnote{ add. information on ARM based solutions and the current trend in wearables: https://www.arm.com/solutions/wearables} or home automation we can find multicore embedded hardware platforms, which garantued high performance \parencite[see][p7]{article15}, network connectivity, security and reliability \parencite[see][p5]{article16}. This field of application is also known as Internet of Things (IoT)\footnote{ for additional information about Internet of Things, please see \parencite[][]{article17}}.

Especially for embedded systems mathematical models as well as numerical solutions, which can be executed both simply and parallel, are suitable. The question arises to what extent parallel execution of different sub-tasks to calculate a problem \parencite[see][p4]{article14} increases the desired cost factor in terms of energy consumption \parencite[see][chapter 3]{inproceedings4} and computational efficiency \parencite[see][p4 Figure 3]{article14}.