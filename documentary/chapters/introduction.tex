Multicore systems are becoming increasingly popular as part of digitization and Industry 4.0 and are playing an increasingly important role in data processing and automation. On the other hand, in addition to efficiency in energy consumption, performance in terms of computation time is required.

Especially for embedded systems mathematical models as well as numerical solutions are suitable, which can be executed both simply and parallel. The question arises to what extent parallel execution of different sub-tasks to calculate a problem increases the desired cost factor in terms of energy consumption and computational efficiency.

In order to develop an optimal solution, the hardware platform must be included in addition to the mathematical model. Only then can suitable prerequisites and characteristics be worked out in order to enable an evaluation of "Parallel Computation Taks on Embedded Multicore Systems".